\begin{thebibliography}{77}
\bibitem{Turk91} M. Turk and A. Pentland.
"Face recognition using eigenfaces."
\emph{Proc. IEEE Conference on Computer Vision and Pattern Recognition.} pp. 586-591.

\bibitem{Viola01}Paul Viola, Michael Jones.
"Robust Real-time Object Detection."
\emph{Second International Workshop on Statistical and Computational Theories of Vision - Modeling, Learning, Computing, and Sampling (Vancouver, Canada)}, July 13, 2001.

\bibitem{Niin10} Niinuma, K., Unsang Park, Fujitsu Labs. Ltd., Kawasaki, Japani, A.K. Jain
"Soft Biometric Traits for Continuous User Authentication" 
\emph{Information Forensics and Security, IEEE Transactions on}, 2010

\bibitem{Jain04}A.K. Jain, S.C. Dass, K. Nandakumar.
"Soft Biometric Traits for Personal Recognition Systems."
\emph{International Conference on Biometric Authentication}, 2004.

\bibitem{Klos00}Andrew J. Klosterman, Gregory R. Ganger.
"Secure Continuous Biometric-Enhanced Authentication", 
Carnegie Mellon University, Tech. Rep. CMU-CS-00-134, 2000.

\bibitem{Jain204}A. K. Jain, S. C. Dass, and K. Nandakumar.
“Can soft biometric traits assist user recognition?”
Proc. SPIE, vol. 5404, pp. 561–572, 2004.

\bibitem{ann99}Anne Adams and Martina Angela Sasse.
Dec,1999/Vol.42.No.12, Communications ofthe ACM,
P40-Parker, D.B. Restating the foundation of information security-- In G.C.
Gable and W.J. Caelli, Eds., IT Security: The Need for International Cooperation.
Elsevier Science Publishers, Holland, 1992

\bibitem{war02} Ware, Karl.
Biometrics and strong authentication McGraw-Hill Osborne

\bibitem{john03}John Woodward, Nicholas M. Orlans, Peter T. Higgins.
Biometrics and strong authentication,McGraw-Hill Inc, 2003

\bibitem{marsav}
Mario Savvides,
http://www.cylab.cmu.edu/education/faculty/savvides.html

\bibitem{bert96}A. Bertillon,
Signaletic Instructions including the theory and practice of Anthropometrical Identification, R.W.
McClaughry Translation, The Werner Company, 1896.

\bibitem{way97}J. L. Wayman,
“Large-scale Civilian Biometric Systems - Issues and Feasibility,” in Proceedings of Card Tech /
Secur Tech ID, 1997.

\bibitem{mon00}
F. Monrose and A. D. Rubin, 
“Keystroke dynamics as biometrics for authentication,” 
Future Generation Comput. Syst., vol. 16, pp.351–359, 2000.

\bibitem{turk03}
A. Altinok and M. Turk, 
“Temporal integration for continuous multi-modal biometrics,”
in Proc. Workshop on Multimodal User Authentication, 2003, pp. 131–137.

\bibitem{sim07}
T. Sim, S. Zhang, R. Janakiraman, and S. Kumar, 
“Continuous verification using multimodal biometrics,” 
IEEE Trans. Pattern Anal. Mach. Intell., vol. 29, no. 4, pp. 687–700, Apr. 2007.

\bibitem{azz08}
A. Azzini, S. Marrara, R. Sassi, and F. Scotti, “A fuzzy approach to
multimodal biometric continuous authentication,” Fuzzy Optimal De-
cision Making, vol. 7, pp. 243–256, 2008.

\bibitem{azz082}
A. Azzini and S. Marrara, “Impostor users discovery using a multi-
modal biometric continuous authentication fuzzy system,” Lecture
Notes in Artificial Intelligence, vol. 5178, pp. 371–378, 2008.

\bibitem{kang06}
H.-B. Kang and M.-H. Ju, “Multi-modal feature integration for secure
authentication,” in Proc. Int. Conf. Intelligent Computing, 2006, pp.
1191–1200.

\bibitem{car03}
C. Carrillo, “Continuous Biometric Authentication for Authorized Air-
craft Personnel: A Proposed Design,” Master’s thesis, Naval Postgrad-
uate School, Monterey, CA, 2003.

\end{thebibliography}
