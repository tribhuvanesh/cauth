%======================================================================
%----------------------------------------------------------------------
%               XX                              X
%                                               X
%               XX    XXX   XXX   XXX      XXX  X  XXXX
%                X   X   X X   X X   X    X   X X X
%                X   XXXXX XXXXX XXXXX    X     X  XXX
%                X   X     X     X     XX X   X X     X
%               XXX   XXX   XXX   XXX  XX  XXX  X XXXX
%----------------------------------------------------------------------
%  	         A SKELETON FILE FOR IEEE PAPER GENERATION
%----------------------------------------------------------------------
%======================================================================

% first, uncomment the desired options:
\documentclass[%
        draft,
        %submission,
        %compressed,
        %final,
        %
        %technote,
        %internal,
        %submitted,
        %inpress,
        %reprint,
        %
        %titlepage,
        notitlepage,
        %anonymous,
        narroweqnarray,
        inline,
        twoside,
        ]{ieee}
%
% some standard modes are:
%
% \documentclass[draft,narroweqnarray,inline]{ieee}
% \documentclass[submission,anonymous,narroweqnarray,inline]{ieee}
% \documentclass[final,narroweqnarray,inline]{ieee}

% Use the `endfloat' package to move figures and tables to the end
% of the paper. Useful for `submission' mode.
%\usepackage {endfloat}

% Use the `times' package to use Helvetica and Times-Roman fonts
% instead of the standard Computer Modern fonts. Useful for the 
% IEEE Computer Society transactions.
% (Note: If you have the commercial package `mathtime,' it is much
% better, but the `times' package works too).
%\usepackage {times}

% In order to use the figure-defining commands in ieeefig.sty...
\usepackage{ieeefig}

\begin{document}

%----------------------------------------------------------------------
% Title Information, Abstract and Keywords
%----------------------------------------------------------------------
\title[Continuous Multimodal User Authentication, using soft and hard biometrics]{%
       Continuous Multimodal User Authentication}

% format author this way for journal articles.
\author[TRIBHUVANESH, SOUMYA AND K.G.SRINIVASA]{%
      Tribhuvanesh Orekondy\member{Student}
      \authorinfo{%
      Department of Computer Science and Engineering,
      M.S.Ramaiah Institute of Technology, Bangalore, India
      Phone: \mbox{}, email: \mbox{tribhuvanesh@gmail.com}}
    \and
      Soumya Gosukonda\member{Student}
      \authorinfo{%
      Department of Computer Science and Engineering,
      M.S.Ramaiah Institute of Technology, Bangalore, India
      Phone: \mbox{}, email: \mbox{soumya.gk@gmail.com}}
    \and
      and Dr.K.G.Srinivasa\member{}
      \authorinfo{%
      Department of Computer Science and Engineering,
      M.S.Ramaiah Institute of Technology, Bangalore, India
      Phone: \mbox{}, email: \mbox{tribhuvanesh@gmail.com}}
  }

% format author this way for conference proceedings
% \author[SHORT NAMES]{%
%       First Author\member{Fellow}
%       \authorinfo{%
%       Department of Electrical Engineering\\
%       Some University, Somewhere CA, 90210, USA\\
%       Phone: (xxx) xxx-xxxx, email: xxx@xxxx.xxx.xxx}
%     \and
%       Second Author\member{Senior Member}
%       \authorinfo{%
%       Department of Electrical Engineering...}
%     \and
%       and Third Author\member{Student Member}
%       \authorinfo{...}
%   }

% specifiy the journal name
% \journal{IEEE Transactions on Something, 1997}

% Or, when the paper is a preprint, try this...
%\journal{IEEE Transactions on Something, 1997, TN\#9999.}

% Or, specify the conference place and date.
% \confplacedate{Ottawa, Canada, May 19--21, 1997}

% make the title
\maketitle               

% do the abstract
\begin{abstract}
Static Authentication methods, although providing a rigid and secure framework to this one-time authentication session, do not provide authentication of the user throughout the session. This leaves the possibility of an imposter gaining access in multiple scenarios.\\
The goal of \emph{Continuous Authentication} is to authenticate the user right from the initial stages of log-in through log-out.
This can be implemented by extrapolating the tried-and-tested static authentication techniques thoughout the session.
But, this introduces new challenges, since these "one-time authentication" techniques are computationally expensive, restricts the user's movement and postures in front of the system, require extra expensive hardware and deviates the user from his work-flow.
In these situations, the user no longer remains uninterrupted by the authentication process in the background.\\
This project proposes Continuous Authentication, using login through conventional passwords followed by authentication using two modes - hard and soft biometrics till logout. The hard biometric trait - facial features, is chosen so that the user need not invest in any additional expensive hardware. The noise inherent in the process of face recognition, is managed by using a machine learning algorithm which captures the temporal data and expresses it as confidence. The soft traits are used in phases when this confidence is high to relieve the CPU of comparatively high computation.
\end{abstract}

% do the keywords
\begin{keywords}
Continuous Authentication, Computer Vision, Face recognition, Machine Learning, Support Vector Machine, Biometrics
\end{keywords}

% start the main text ...
%----------------------------------------------------------------------
% SECTION I: Introduction
%----------------------------------------------------------------------
\section{Introduction}
\PARstart{A}{}uthentication of the user is an important aspect of security that has been implemented over time, in many ways based on the following aspects of a user \cite{Klos00}:
\begin{itemize}
	\item Something the user \emph{knows}: This refers to conventional password based login. Passwords are encrypted and stored in the database.
	\item Something the user \emph{has}: This can refer to tokens that a user possesses which authenticate a user, like an ID card. 
	\item Something the user \emph{is}: This refers to biological or behavioural traits of a user such as fingerprint, facial features or Keystroke Dynamics\cite{mon00}. These traits are enrolled during account creation or during log-in (depending on what type of characteristic feature is being recorded) and are used when the system needs to re-inforce its belief of the user's identity.
\end{itemize}
All of the above have been used in various combinations, for static as well as continuous authentication of a user. \emph{Static Authentication} is the method of authenticating a user, at the time of log-in. Password-based authentication are a popular implementation of this concept. However, passwords can be subjected to being stolen or cracked using various techniques. Also, static authentication methods, in general, do not ensure that the user remains authenticated throughout the session. Therefore, there is need for a system that continually verifies the user's identity, while allowing the user to continue his/her work uninterrupted. Such a system is called a \emph{Continuous Authentication system}.
A lot of research and experimentation has been done in the field of Continuous Authentication \cite{Niin10,Klos00,mon00,turk03,sim07,azz08,azz082}, though it has not been adopted for widespread usage as readily as the Static Authentication systems have been. Continuous Authentication tends to utilize a user's biometric traits for identification, since they can be monitored without distracting the user from his/her work \cite{Klos00}. \emph{Biometric traits} refer to the physiological or behavioural traits of a user that can identify a user, for a session. These traits can be divided into the following two categories:
\begin{itemize}
	\item Hard Biometric traits: These are physical traits of a user that are assumed to be present universally and can uniquely identify an individual. For example, fingerprints, facial features, DNA and so on.
	\item Soft Biometric traits: These are characteristics of a user that "provide some information about the individual, but lack the distinctiveness and permanence to sufficiently differentiate any two individuals"\cite{Jain204}. For example, colour of clothing/skin/eye/hair, gender and other such factors.
\end{itemize}

The issues involved with the integration of biometric-enhanced authentication systems have been well-researched by \emph {Klosterman et al.}\cite{Klos00}. They point out that for unobtrusive continuous monitoring of a user's biometric traits, the trait cannot be something that needs to be in contact with a sensor. Therefore, facial features of a user make for an ideal choice for continuous unobtrusive monitoring. Another important observation in the previously cited paper is the fact that biometrics are expensive to compute. In case of facial features, the image processing and recognition algorithms can be computationally much more expensive as compared to password verification. Therefore, we intend to incorporate Soft Biometric Authentication in our system, which possesses the qualities of being computationally inexpensive and unobtrusive.

The feasibility of using soft biometric traits for user identification was researched by \emph{Jain et al.}\cite{Jain204}, and it was found to reduce computational costs of recognition. A more detailed implementation of authentication incorporating soft biometric traits has been researched into by \emph{Niinuma et al.}\cite{Niin10}. They create a template of the user's shirt and skin colour and generate "similarity scores" for the subsequent frames in which the user is captured. The user is identified based on the comparison of the similarity scores with a certain threshold value. While this model yields fairly good results, according to the results, it leaves scope for improvement of recognition by considering temporal information. By temporal information, we mean, information about the user's identity considered over a certain time period. 

We propose a multimodal continuous system, wherein, a user, upon entering the right username-password combination, will be checked for their facial features (Hard Biometric Authentication). If this is successful then the user is said to be logged in, and then is enrolled for his/her Soft Biometric traits, namely the user's shirt colour. This is continually checked in the subsequent frames, until the user leaves the workstation, or if the shirt colour detected is not the same as that of the user enrolled previously. During Hard Biometric Authentication we intend to recognize the user based on temporal information about the user's identity.

\section{Motivation}


\section{Related Work}

\section{Contribution}

\section{Architecture of proposed work}

\section{Algorithms and techniques used}

\section{Implementation}

\section{Results}

\section{Conclusion}

% do the biliography:
\bibliographystyle{IEEEbib}
\bibliography{my-bibliography-file}

% where ``my-bibliography-file.bib'' is the name of the file with all the 
% BibTeX entries.

% do the biographies...
\begin{biography}{Gregory L. Plett}
  A bio with no face...
\end{biography}

% If you want a picture with your biography, then specify the name of
% the postscript file in square brackets. That is, uncomment the
% following three lines and change the name of "face.ps" to the name of 
% your file.
%\begin{biography}[face.ps]{Gregory L. Plett}
%  A bio with a face...
%\end{biography}

%----------------------------------------------------------------------
% FIGURES
%----------------------------------------------------------------------
% There are many ways to include figures in the text. We will assume
% that the figure is some sort of EPS file.
%
% The outdated packages epsfig and psfig allow you to insert figures
% like: \psfig{filename.eps} These should really be done now using the
% \includegraphics{filename.eps} command.  
%
% i.e.,
%
% \includegraphics{file.eps}
%
% whenever you want to include the EPS file 'file.eps'. There are many
% options for the includegraphics command, and are outlined in the
% on-line documentation for the "graphics bundle". Using the options,
% you can specify the height, total height (height+depth), width, scale,
% angle, origin, bounding box "bb",view port, and can trim from around
% the sides of the figure. You can also force LaTeX to clip the EPS file
% to the bounding box in the file. I find that I often use the scale,
% trim and clip commands.
% 
% \includegraphics[scale=0.6,trim=0 0 0 0,clip=]{file.eps}
% 
% which magnifies the graphics by 0.6 (If I create a graphics for an
% overhead projector transparency, I find that a magnification of 0.6
% makes it look much better in a paper), trims 0 points off
% of the left, bottom, right and top, and clips the graphics. If the
% trim numbers are negative, space is added around the figure. This can
% be useful to help center the graphics, if the EPS file bounding box is
% not quite right.
% 
% To center the graphics,
% 
% \begin{center}
% \includegraphics...
% \end{center}
% 
% I have not yet written good documentation for this, but another 
% package which helps in figure management is the package ieeefig.sty,
% available at: http://www-isl.stanford.edu/people/glp/ieee.shtml
% Specify:
% 
%\usepackage{ieeefig} 
% 
% in the preamble, and whenever you want a figure,
% 
\figdef{filename}
% 
% where, filename.tex is a LaTeX file which defines what the figure is.
% It may be as simple as
% 
% \inserteps{filename.eps}
%
% or
% \inserteps[includegraphics options]{filename.eps}
% 
% or may be a very complicated LaTeX file. 

\end{document}
