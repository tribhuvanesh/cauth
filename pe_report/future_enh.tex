\documentclass[12pt]{article}			% For LaTeX 2e
						% other documentclass options:
						% draft, fleqn, openbib, 12pt

\usepackage{graphicx}	 			% insert PostScript figures
\usepackage{caption}
\usepackage{subcaption}
\usepackage{wrapfig}
\usepackage{amsmath}
%% \usepackage{setspace}   % controllabel line spacing
%% If an increased spacing different from one-and-a-half or double spacing is
%% required then the spacing environment can be used.  The spacing environment 
%% takes one argument which is the baselinestretch to use,
%%         e.g., \begin{spacing}{2.5}  ...  \end{spacing}


% the following produces 1 inch margins all around with no header or footer
\topmargin	=10.mm		% beyond 25.mm
\oddsidemargin	=0.mm		% beyond 25.mm
\evensidemargin	=0.mm		% beyond 25.mm
\headheight	=0.mm
\headsep	=0.mm
\textheight	=220.mm
\textwidth	=165.mm
					% SOME USEFUL OPTIONS:
% \pagestyle{empty}			% no page numbers
 \parindent  15.mm			% indent paragraph by this much
 \parskip     2.mm			% space between paragraphs
% \mathindent 20.mm			% indent math equations by this much

\newcommand{\MyTabs}{ \hspace*{25.mm} \= \hspace*{25.mm} \= \hspace*{25.mm} \= \hspace*{25.mm} \= \hspace*{25.mm} \= \hspace*{25.mm} \kill }

\graphicspath{{../Figures/}{../data/:}}  % post-script figures here or in /.

					% Helps LaTeX put figures where YOU want
 \renewcommand{\topfraction}{0.9}	% 90% of page top can be a float
 \renewcommand{\bottomfraction}{0.9}	% 90% of page bottom can be a float
 \renewcommand{\textfraction}{0.1}	% only 10% of page must to be text

\linespread{1.2}
\alph{footnote}				% make title footnotes alpha-numeric

\begin{document}			% REQUIRED

\begin{center}
	{\LARGE \bf Future Enhancements}

\end{center}
\noindent{}Need for biometric authentication arose from the fact that other security measures such as password and/or a smart ID card are prone to theft or loss. Biometrics depend on utilising features a user already possesses, that can uniquely identify a user atleast for a given session. However, using biometric identification has its downfalls. Since the characteristic of biometric traits is that they are not secretive, unlike a password, they can be gathered anywhere by an imposter and reconstructed for the authentication system to gain access. For example, a fingerprint left behind on some surface, can be picked up by imposters and reconstructed in front of the sensors. Even for face recognition systems, it is possible to fool the system by showing a photograph of the user or manipulating the video feed of the system. Countermeasures exist for such instances, such as, 3d reconstruction of face from more than two cameras or ensuring that the video feed cannot be manipulated. However, these measures make it a costly solution. Therefore, it can be concluded that biometric mode of authentication should be used more as a support system along with password/ smartID systems, to strengthen the authentication process, rather than a standalone mode of authentication.   

\noindent{}The following can be implemented in the future to enhance this system:
\begin{itemize}
\item Support for multiple users sharing a certain account; this may require a "biometric handoff" to occur between users.
\item Improve accuracy of face recognition under varied lighting conditions by implementing recognition using a different approach such as fuzzy logic.
\item Make provision for recognizing any kind of tampering occuring to the video feed, so as to prevent authenticating imposters. This can be done by restricting access to the webcam feed via parameters that define access to it.
\end{itemize}
\end{document}
